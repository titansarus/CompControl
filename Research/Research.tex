\documentclass[conference]{IEEEtran-ModifiedForMVIP}
% این فایل از روی نمونه‌ فایل ارائه شده برای کنفرانس مهندسی برق ایران ICEE 
% برای کنفرانس بینایی ماشین و پردازش تصویر ایران به‌روزرسانی شده است.
% فایل منبع و فایلهای ضمیمه‌ی آن با تغییر فایلهایی که توسط آقای دکتر محمود امین طوسی (دانشگاه
% حکیم سبزواری، http://profs.hsu.ac.ir/mamintoosi) در سایت www.parsilatex.com قرار داده 
% شده بودند به دست آمده است. این تغییرات توسط دکتر مسعود بابایی‌زاده داده شده است. 
% البته فایل IEEEtran-ModifiedForICEE.cls در اینجا، 
% اصلاح شده فایل آقای دکتر امین طوسی نیست و مستقیما با دستکاری
% در فایل IEEEtran.cls توسط مسعود بابایی زاده ایجاد شده است.
% برای کنفرانس بینایی ماشین و پردازش تصویر ایران، فایل IEEEtran-ModifiedForICEE.cls 
% با نام IEEEtran-ModifiedForMVIP.cls  تغییر داده شده است.

% مقاله اصلی که این فایل از تغییر فایل آن به دست آمده است، در هفدهمین کنفرانس مهندسی برق 
% ایران در اردیبهشت ۸۸ ارائه شده بوده است.

% شما می‌توانید از این فایل به عنوان یک الگو برای مقالات خود استفاده نمایید.

% برای پردازش پس از یکبار استفاده از xelatex با استفاده از دستور زیر لیست مراجع را تولید نمایید:
% bibtex MVIP_FA_LaTeX_SamplePaper
% و سپس دوبار استفاده از xelatex. 

%%%% فراخوانی پکیج‌های مورد نیاز کاربر %%%%%%%%%%%%%%%%%%%%%%%%%%%%%%%%%%%%%%%%%%%%%%%%%%
\usepackage{setspace}
\usepackage{subfigure}
\usepackage{algorithm}
\usepackage{algorithmic}
\usepackage{graphicx}
\usepackage{amsmath}
%\usepackage[colorlinks, citecolor=blue]{hyperref}

%%%%%%%%%%%%%%%%%%%%%%%%%%%%%%%%%%%%%%%%%%%%%%%%%%%%%%%%%%%%%%%%%%%%%%%%%%%%%%%%%%%%%%%%
%%%% فراخوانی تنظیمات مورد نیاز کنفرانس مهندسی برق ایران و پکیج زی‌پرشین %%%%%%%%%%%%%%%%
\input{MVIP_Settings1.tex}
% یکی از دو روش زیر را انتخاب کنید. گذاشتن حالت «کشیده» باعث می‌شود که تنطیم
% طول خطها بجای اینکه با کم و زیاد کردن فاصله بین کلمات انجام شود، با کشیدن
% کلمات انجام شود. این حالت در فارسی صحیح‌تر و خیلی زیباتر است (برخلاف انگلیسی که کشیدن کلمات
% در آن معنی ندارد و تنطیم طول خطوط فقط با کم و زیاد کردن فاصله بین کلمات صورت
% می‌گیرد). اما با استفاده از حالت «کشیده»، اگر از Acrobat Adobe برای دیدن خروجی پی‌دی‌اف
% استفاده کنید این کشیده‌ها را می‌بینید که چندان زیبا نیست (در نسخه چاپی وجود ندارند).
% اگر می‌خواهید اینها را نبینید در قسمت  Edit->Preferences->PageDisplay گزینه
% Enhance Thin Lines
% را غیرفعال کنید. اما اگر از SumatraPDF برای دیدن فایل پی‌دی‌اف استفاده می‌کنید، تنظیم خاصی
% نیاز نیست.
\usepackage[Kashida]{xepersian}
% \usepackage{xepersian}
\input{MVIP_Settings2.tex}

%%%%%%%%%%%%%%%%%%%%%%%%%%%%%%%%%%%%%%%%%%%%%%%%%%%%%%%%%%%%%%%%%%%%%%%%%%%%%%%%%%%%%%%%
% تعریف دستورات جدید مورد نیاز کاربر %%%%%%%%%%%%%%%%%%%%%%%%%%%%%%%%%%%%%%%%%%%%%%%%%%%
\newcommand\femph[1]{\lr{''}#1\lr{``}}
\newcommand{\SR}{وضوحِ برتر}%{\textiranic{ وضوحِ برتر }}
\newcommand{\HR}{وضوح بالا}
\newcommand{\registration}{ثبت تصویر}
\newcommand{\fusion}{آمیختن}
\newcommand{\fused}{آمیخته}

\newcommand{\warp}{\mathbf{W}(\mathbf{x};\mathbf{p})}
\newcommand{\IWarp}{I(\mathbf{W}(\mathbf{x};\mathbf{p}))}
\newcommand{\round}[2]{\frac{\partial{#1}}{\partial{#2}}}
\newcommand{\roundB}[2]{\frac{\partial{\mathbf{#1}}}{\partial{\mathbf{#2}}}}


% شروع متن اصلی %%%%%%%%%%%%%%%%%%%%%%%%%%%%%%%%%%
\begin{document}
% دستور زیر باعث امکان استفاده از \thanks می‌شود.
\IEEEoverridecommandlockouts 

\title{
	حسگرهای مبتنی بر اثر هال 
}

\author{
\IEEEauthorblockN{
	امیرمهدی نامجو}
\IEEEauthorblockA{
دانشگاه صنعتی شریف، دانشکده مهندسی کامپیوتر\\
{شماره دانشجویی: 97107212}\\}
\IEEEauthorblockA{
	درس اندازه‌گیری و کنترل کامپیوتری \\
	{استاد گرامی: جناب آقای دکتر همت‌یار}\\}
}







%\date{}
\maketitle

\begin{abstract}
در این گزارش، به حسگرهای مبتنی بر اثر هال می‌پردازیم. این حسگرها، جزو حسگرهای غیرتماسی هستند. در ابتدا به طول خلاصه به حسگر‌های غیرتماسی پرداخته و پس از تببین کلی موضوع، به بررسی حسگر‌های مبتنی بر اثر هال می‌پردازیم. در ابتدا به تشریح اثر هال و سپس نحوه استفاده از آن به عنوان حسگر و کاربردهای آن در قسمت‌های مختلف و همچنین مزایا و معایب استفاده از این حسگرها می‌پردازیم.
 \end{abstract}
\begin{IEEEkeywords}
حسگرهای غیرتماسی - اثر هال - حسگرهای مبتنی بر اثر هال
\end{IEEEkeywords}

%\IEEEpeerreviewmaketitle

\section{حسگرهای غیرتماسی}
\subsection{تعریف}
حسگرهای غیرتماسی، گونه‌ای از حسگرها هستند که با فناوری‌های گوناگون، بدون نیاز به تماس مستقیم، مقدار مد نظر را اندازه گیری‌ می‌کنند. در این نوع حسگر‌ها اصطکاک و اجزای متحرک نقش اساسی ایفا نکرده و در نتیجه فرسودگی کاهش‌ می‌یابد.

حسگرهای تماسی، در نقطه مقابل حسگرهای تماسی هستند. در حسگرهای تماسی نیاز به تماس مستقیم فیزیکی برای اندازه‌گیری متغیر مدنظر وجود دارد. این تماس می‌تواند از طریق جا به جایی یک پیستون، جابه‌جایی یک رسانا روی رسانای دیگر (نظیر پتانسیومتر) و موارد دیگر باشد؛ اما به هر حال اصل نیاز به جا‌به‌جایی و تماس مستقیم در ساختار خود حسگر نقش اساسی دارد.
\cite{varriohm_2020}
 

\subsection{دلایل عمومی استفاده از حسگر‌های غیرتماسی}
حسگر‌های غیرتماسی، مزایای بالایی دارند که باعث شده امروزه با گسترش تکنولوژی و کاهش هزینه ساخت آن‌ها، شاهد افزایش اقبال عمومی نسبت به آنان هستیم. از جمله این دلایل، می‌توان به موارد زیر اشاره کرد. \cite{varriohm_2020}

\begin{enumerate}
	\item طول عمر بیش‌تر
	\item نرخ پاسخگویی بالاتر
	\item قابلیت اطمینان
	\LTRfootnote{Reliability} بالاتر
	\item قابلیت اتکا بالاتر
	\item کارایی پیوسته و بیش‌تر (بدون استهلاک)
	\item مقاومت بیش‌تر نسبت به گرد‌ و خاک

	
\end{enumerate}


\subsection{نمونه‌هایی از حسگرهای غیرتماسی}

حسگرهای غیرتماسی، به شکل‌های مختلف و برای امورمختلفی ساخته شده‌اند. بعضی از انواع این حسگرها به شرح زیر است:

\begin{enumerate}
	\item \lr{LVDT}: مبدل تفاضلی متغیر خطی یا \lr{LVDT}، گونه‌ای از حسگر‌های غیرتماسی است  که  کارکرد آن -همان طور که در درس اندازه‌گیری و کنترل کامپیوتری دیدیم- مبتنی بر جریان القایی، سیم‌پیچ‌های فلزی و هسته فلزی است. در LVDT ها از حداقل دو سیم‌پیچ فلزی استفاده می‌شود. 
	\cite{varriohm_2020}
	
	\item \lr{RVDT}: مبدل تفاضلی متغیر زاویه‌ای یا \lr{RVDT} گونه‌ای دیگر از حسگر‌های تماسی است که کارکردی شبیه \lr{LVDT} دارد ولی مبنای حرکتی آن به صورت حرکت‌های چرخشی است.
	\cite{varriohm_2020}
	
	\item
	\lr{PIPS}:
	این حسگر‌ها که تکنولوژی انحصاری شرکت Positek هستند، کارکردی شبیه LVDT ها دارند ولی برخلاف LVDT ها، در این نوع حسگر‌ها از یک سیم پیچ استفاده می‌شود.
	\cite{varriohm_2020}
	
	\item
	حسگرهای فراصوتی: حسگرهای فراصوتی
	\LTRfootnote{Ultrasonic}
	از امواج صوتی با فرکانس بالا برای کار استفاده می‌کنند. مثلا برای تشخیص فاصله، موجی به سمت هدف فرستاده شده و از مدت زمانی که طول می‌کشد تا انعکاس این موج دوباره به حسگر برسد، برای بدست آوردن فاصله استفاده می شود. از این حسگر‌ها عموما برای اندازه‌گیری‌های فواصل طولانی استفاده می شود ولی امکان اندازه‌گیری فواصل کوتاه‌تر نظیر عمق یک مایع در مخزن هم به کمک آنان وجود دارد. \cite{edwards_2017}
	
	\item دماسنج‌های تابشی:
	هر جسمی با دمای بالاتر از صفر کلوین، تابش‌های گرمای دارد که به کمک حسگر‌های خاص، قابل اندازه‌گیری هستند. این حسگر‌ها عموما نقشه‌ای یک یا دوبعدی از توزیع دما در نقاط مختلف یک محیط را براساس تابش‌های دریافتی در اختیار کاربر قرار می‌دهند. یکی از رایج‌ترین کاربردهای آنان، در اندازه‌گیری دمای بدن انسان به شکل سریع است. \cite{edwards_2017}
	
		\item حسگر‌های مبتنی بر جریان گردابی
	\LTRfootnote{Eddy Current}:
	این حسگر‌ها که از به نوعی مبتنی بر القای الکرتیکی هستند، از طریق میدان‌های مغناطیسی ایجاد شده در اثر جریان متناوب و تغییر جهت این میدان‌های مغناطیسی و جریان القایی در اثر این تغییر جهت، موقعیت اجسام را تشخیص می‌دهند. این حسگر‌ها معمولا در ابعاد کوچک ساخته شده و برای کاربرد‌های مکانی ریزمقیاس‌تر نظیر تنظیم کردن ماشین‌ابزار، اندازه‌گیری لرزش اجسام و... استفاده می‌شود.	\cite{ixthus_instrumentation_what_nodate}
	
	\item 
	حسگر‌های نوری فاصله‌ای: از تعدادی از حسگر‌های نوری هم مشابه حسگرهای فراصوتی برای تشخیص فاصله یا جابه‌جایی از طریق اندازه‌گیری شدت و زمان نور بازتابی از سطوح مختلف استفاده می شود. البته گاهی اوقات نوع جنس سط و یکسری ویژگی‌های آن می‌تواند بر کارکرد این سنسورها اثرگذار باشد و در نتیجه باید در زمینه استفاده از آنان ماطلعه کافی داشت.
	\cite{edwards_2017}
	
		

	
	\item
	حسگر‌های مبتنی بر اثر هال:
	این حسگر‌ها با تکیه بر اثر هال که یک پدیده الکترومغناطیسی است و بر مبنای اختلاف ولتاژ ایجاد شده در اثر قرار گرفتن یک صفحه فلزی حامل جریان در میدان مغنطیسی، کار می‌کنند. این حسگر‌ها را به طور مفصل‌تری در این گزارش بررسی می‌کنیم.
	
	
	
	
\end{enumerate}


\section{اثر هال}

اثر هال، پدیده‌ای الکترومغناطیسی است که اولین بار در سال ۱۸۷۹ میلادی توسط ادوین هال
\LTRfootnote{Edwin Hall}
، کشف و گزارش شد و از این رو به افتخار نام این دانشمند، این نام بر روی آن قرار گرفتها است. \cite{noauthor_hall_2021}

اثر هال مبتنی بر ذات جریان الکتریکی است. جریان الکتریکی از جا به جایی حاملان بار در یک رسانا اتفاق می‌افتد. این ذرات حامل بار، الکترون‌ها و یون‌ها هستند و البته در عمل، گاهی‌ اوقات به جای الکترون، مفهوم جا به جایی حفره‌هایی با بار مثبت هم مطرح می‌شود. این ذرات بردار در صورتی که در حضور یک میدان مغناطیسی  قرار بگیرند، تحت تاثیر نیروی لورنتز خواهند بود. در صورت نبود میدان مغناطیسی، این ذرات باردار در مسیری تقریبا مستقیم در رسانا که اندکی به دلیل ناخالصی‌ها ممکن است جا به جا شود، حرکت می‌کنند.

برای یک قطعه فلزی ساده که تنها الکترون‌ها در آن حرکت می‌کنند، می‌توان به راحتی از روی رابطه نیروی لورنتز‌، این پدیده را توجیه کرد. برای این منظور باید به شکل \ref{fig:hall} توجه کرد.



\begin{figure}[t]

	\centering 
	\includegraphics[width=70mm]{Images/fig1.png}
	\caption{نمودار کلی اثر هال 
\cite{noauthor_hall_2021}	
}\label{fig:hall}
\end{figure}

فرمول نیروی لورنتز به صورت 
\begin{equation}
	\boldsymbol{F} = q(\boldsymbol{E} + \boldsymbol{v} \times \boldsymbol{B})
\end{equation}
است.
\cite{halliday2010fundamentals}

با فرض سرعت در راستای $x$ و میدان مغناطیسی در راستای $z$ می‌دانیم که عبارت $v_x B_z$ به صورت منفی ظاهر خواهد شد. در حالتی که نیروی وارده صفر باشد، رابطه
\begin{equation}
0 = E_y - v_x B_z
\label{eq:2}
\end{equation}
را خواهیم داشت. البته باید توجه داشت که عملا به ازای الکترون‌ها،
$v_x\rightarrow -v_x$
و
$q \rightarrow -q$
است. $E_y$ همان میدان‌‌الکتریکی القایی است که منجر به ایجاد ولتاژ القایی اثرهال می‌شود.  در نتیجه از آن‌جایی که
$E_y = \frac{-V_H}{w}$
در نتیجه با جایگزینی در عبارت \eqref{eq:2} به رابطه
\begin{equation}
	V_H = v_x B_z w
	\label{eq:3}
\end{equation}
می‌رسیم.

با این وجود جریان قراردادی که عملا جریان حفره‌های حامل بار مثبت است، در خلاف جهت جریان الکترون‌ها و با بار منفی است، در نتیجه می‌توانیم برای جریان به رابطه
\begin{equation}
	I_x = ntw(-v_x)(-e)
\end{equation}
برسیم که در آن $n$ چگالی تعداد حاملین بار با واحد $m^{-3}$ است و $tw$ هم سطح مقطع عبوری را مشخص می‌کند. با حل معادله برحسب $w$ وجایگذاری آن در
\eqref{eq:3}
داریم \cite{noauthor_hall_2021}:
\begin{equation}
	V_H = \frac{I_x B_z}{n t e}.
\end{equation}


البته رایج است که در این رابطه ضریبی تحت نام ضریب هال به صورت 
\begin{equation}
	R_H = \frac{1}{n e}
\end{equation}
با واحد $m^3/C$ یا $\Omega cm/G$ تعریف کنند و رابطه نهایی به صورت
\begin{equation}
	V_H = R_H (\frac{I B}{t})
	\label{eq:7}
\end{equation}
نمایش داده می‌شود. در نتیجه عوامل اصلی در تعیین ولتاژ، شدت جریان، ضخامت ورقه و میدان مغناطیسی است.
\cite{ele_hall_2013}

نکته مهمی که در این روابط وجود دارد، این است که عملا جنس ذره حامل بار در آن اثر دارد. یعنی این که حامل جریان را الکترون فرض کردیم، در روابط بدست آمده اثرگذار بود. در نتیجه اگر فرض کنیم جهت میدان مغناطیسی برعکس شود، اثر برعکس بر الکترون‌ها گذاشته و منجر به تغییر علامت ولتاژ القایی می‌شود و از این طریق، می‌توانیم در جهت میدان مغناطیسی نیز تمایز قائل شویم.
\cite{noauthor_hall_2021}

نکته حائز اهمیت دیگر این است که  در عمل، بیش‌تر اوقات برای ورقه، از یک نیم‌رسانا استفاده می‌شود. دلیل این موضوع هم تاثیر گذاری همزمان الکترون‌ها و حفره‌ها در نیم‌رسانا است که باعث می‌شود ضرایب $R_H$ بزرگ‌تری بدست آمده و آشکارسازی ولتاژ به شکل راحت‌تری صورت بگیرد. برای نیم‌رسانا‌ها می‌توان در یک تقریب نسبتا خوب، $R_H$ را به صورت
\begin{equation}
	R_H = \frac{p \mu_h^2 - n \mu_e^2}{e(p\mu_h + n \mu_e)^2}
\end{equation}
بدست آورد که در آن، $n$ ضریب تجمع الکترون‌ها، $p$ ضریب تجمع حفره‌ها، $\mu_e$ تحرک‌پذیری الکتریکی الکترون‌ها و $\mu_h$ تحرک‌پذیری الکتریکی حفره‌ها و $e$ بار الکترون است. البته در چنین مواردی، حلیل نهایی تمامی روابط موجود به سادگی رابطه \eqref{eq:7} نیست اما آن رابطه می‌تواند دیدی سطح بالا و کلی از نحوه اثرگذاری پدیده هال به ما بدهد.
\cite{noauthor_hall_2021}


اگر بخواهیم به شکل  شهودی و مستقل از روابط ریاضیاتی به اثر هال نگاه کنیم، می‌توانیم آن را این طور توجیه کنیم که در اثر قرار گرفتن در میدان مغناطیسی، الکترون‌ها به جای عبور یک نواخت از نوار رسانا (نیم‌رسانا)، در یک سمت آن تجمع بیش‌تری پیدا کرده و در نتیجه آن حفره‌های مثبت در سمت دیگر تجمع خواهند کرد و به همین دلیل، اختلاف ولتاژ بین دو قسمت رسانا (نیم‌رسانا) ایجاد خواهد شد.


%\lr{SSIM}\LTRfootnote{ Structural SIMilarity (SSIM)}
%%%%%%%%%%%%%%%%%%%%%%%%%%%%%%% \section{شیوه‌ی پیشنهادی و نتایج}
\section{شیوه‌ی پیشنهادی}\label{Sec:TheProposedMethod}
 در شیوه‌ی پیشنهادی برای وضوح برتر توسط نگارندگان در \cite{Amintoosi08reconstruction}، هر یک از تصاویر باوضوح بالا، به عنوان تصویر آموزشی، متناظر با قسمتی از تصویرِ باوضوح پایین هستند.  تصاویر آموزشی می‌توانند تفاوتهایی با تصویر اصلی از نقطه نظر شدت روشنائی یا زاویه‌ی اخذ داشته باشند. 
 این تفاوتها می‌تواند ناشی از برداشت عکسها در زمانهای متفاوت و یا با دوربینهای متفاوت و از زوایای مختلف باشد. در این شیوه ابتدا تصویر با وضوح پایین به اندازه‌ی مطلوب بزرگ شده و سپس  تبدیل مناسبی برای نگاشت هر یک از تصاویر آموزشی بر روی تصویر مورد نظر با استفاده از  نقاط کلیدی \lr{SIFT}\LTRfootnote{ Scale Invariant Feature Transform (SIFT)} و الگوریتم \lr{RANSAC}\LTRfootnote{ RANdom SAmple Consensus (RANSAC)} در قالب ماتریس هوموگرافی
%\LTRfootnote{ Homography matrix}
 پیدا می‌شود. در انتها تصویرِِ باوضوح بالای نگاشت شده، با تصویر باوضوح پایین ورودی
% \fused\LTRfootnote{ Fused}
ترکیب می‌شود.
 چارچوب کلی کار در این مقاله در شکل~\ref{fig:OneLR_oneHR} آمده است. دو مرحله‌ی «دقیق‌تر نمودن مدل با استفاده از ثبت تصویر مبتنی بر ناحیه» و «همرنگ نمودن تصاویر در نواحی مرزی» در این مقاله اضافه شده‌اند. از آنجا که ذکر روش کار برای یک یا چند تصویر آموزشی تفاوتی ندارد، در اینجا فرض بر آن است که فقط از یک تصویر آموزشی استفاده می‌شود.

%------------------------- Visio Chart, OneLR_OneHR
\begin{figure}[t]
%\centerline{\XeTeXpdffile "Images/OneLR_oneHR.pdf" height 14cm}
\centering 
\includegraphics[width=70mm]{Images/OneLR_oneHR.pdf}
\caption{چارچوب کلی شیوه‌ی پیشنهادی.}\label{fig:OneLR_oneHR}
\end{figure}


مهمترین قسمت در کار حاضر استفاده از معیار مقایسه‌ی ساختاری دو تصویر (\lr{SSIM}) برای بهبود شیوه‌ی ثبت تصویر لوکاس-کاناد~\cite{Lucas81iterative} می‌باشد. در مراجع از فرمولبندی‌های متفاوتی برای بیان این شیوه استفاده شده است. در این مقاله از فرمولبندی ذکر شده در \cite{Baker04lucas-kanade20part1} استفاده خواهیم نمود و لذا مروری بر این فرمولبندی ضروری می‌باشد که در ادامه ذکر خواهد شد. پس از آن نگاهی بر معیار مقایسه‌ی \lr{SSIM} داشته و سپس روش پیشنهادی بر اساس آنها بیان خواهد شد.

\subsection{الگوریتم لوکاس-کاناد}
 هدف در شیوه‌ی ثبت تصویر لوکاس-کاناد~\cite{Lucas81iterative} کمینه‌سازی مجموع مربع تفاضلات زیر بین تصویر آموزشیِ $T(\mathbf{x})$ و نگاشت تصویر ورودیِ $I(\mathbf{x})$ است:
 \begin{equation}\label{eq:SSD_L2Norm}
    \textrm{SSD}=\sum_x[\IWarp-T(\mathbf{x})]^2
\end{equation}
که در آن $\warp$ بیانگر مدل تبدیل‌ (در اینجا پروجکتیو)، $\mathbf{p}=(p_1,\dots,p_8)^T$ پارامترهای مدل تبدیل، $\IWarp$ نگاشت تصویر ورودی $I$ بر روی مختصات تصویر آموزشی $T$ و $\mathbf{x} =(x,y)^T$ مختصات یک پیکسل می‌باشد.
 کمینه‌سازی \eqref{eq:SSD_L2Norm} نسبت به $\mathbf{p}$ انجام می‌شود. 
در شیوه‌ی لوکاس-کاناد فرض بر آن است که در ابتدا تخمینی از مدل دردست بوده و در یک فرآیند تکراری این تخمین بهبود داده می‌شود؛
در هر دور ابتدا عبارت زیر بر اساس $\triangle\mathbf{p}$ کمینه شده:
\begin{equation}\label{eq:SSD_L2Norm_deltap}
    \sum_x[I(\mathbf{W}(\mathbf{x;\mathbf{p+\triangle p}}))-T(\mathbf{x})]^2
\end{equation}
 و سپس پارامترها بروزرسانی می‌شوند:
\begin{equation}
    \mathbf{p}\leftarrow\mathbf{p+\triangle p}
\end{equation}
دو مرحله‌ی فوق تا مادامیکه الگوریتم همگرا نشده است تکرار خواهند شد. در فرآیند کمینه‌سازی، $\mathbf{\triangle p}$ به صورت زیر محاسبه می‌شود:
\begin{equation}\label{eq:deltap}
    \triangle\mathbf{p} = H^{-1} \sum_x[\nabla I\roundB{W}{p}]^T[T(\mathbf{x})-\IWarp]
\end{equation}
که در آن $H$، ماتریس هسین تقریبی\LTRfootnote{ Approximate Hessian Matrix}، به صورت زیر بدست می‌آید:
\begin{equation}\label{eq:Hessian}
    H = \sum_x[\nabla I\roundB{W}{p}]^T[\nabla I\roundB{W}{p}]
\end{equation}
این مراحل در الگوریتم \ref{alg1} نشان داده شده است \cite{Baker04lucas-kanade20part1}.
گونه‌های مختلفی از این الگوریتم پیشنهاد شده‌اند. سلزکی\LTRfootnote{ Szeliski} در \cite{Szeliski96video} از روش بهینه‌سازی لونبرگ-مارکورت برای قسمت بهینه سازی آن استفاده نموده است که اساس کار ما در بخش‌های آتی می‌باشد.
\begin{algorithm}[t]
\caption{الگوریتم ثبت تصویر لوکاس-کاناد مبتنی بر بهینه‌سازی گوس-نیوتون \lr{(LK-GN)}.} \label{alg1}
\singlespacing
\begin{latin}
\textbf{Input}:
The reference image $I$ and template image $T$.\\
\textbf{Output}: Reg. parameters
$\mathbf{p}=(p_1,\dots,p_n)^T$ as the warp model $\warp$.
\begin{algorithmic}[1]
\REPEAT
  \STATE Warp $I$ with $\warp$ to compute $\IWarp$. 
  \STATE Compute the error image $T(x)-\IWarp$ 
  \STATE Warp the gradient $\nabla I$ with $\warp$. \STATE Evaluate the Jacobian
    $\roundB{W}{p}$ at $(\mathbf{x;p})$. 
  \STATE Compute the steepest descent images $\nabla I\roundB{W}{p}$. 
  \STATE \label{line:Hessian} Compute the Hessian matrix using Equation
    \eqref{eq:Hessian}. 
  \STATE Compute $[\nabla I\roundB{W}{p}]^T$ and $[T(x)-\IWarp]$ 
  \STATE \label{alg1:deltap} Compute $\triangle\mathbf{p}$ using Equation \eqref{eq:deltap} 
  \STATE Update the parameters $\mathbf{p}\leftarrow\mathbf{p}+\triangle\mathbf{p}$ 
\UNTIL{$||\triangle\mathbf{p}||\leq\epsilon$ or Reaching to Maximum Iteration allowed}
\end{algorithmic}
\end{latin}
\end{algorithm}


\subsection{ارزیابی خطا با محک \lr{SSIM}}
در \cite{Wang04image} محک \lr{MSSIM}\LTRfootnote{ Mean Structural SIMilarity} برای اندازه‌گیری کیفیت یک تصویر، به صورت زیر تعریف شده است:
\begin{equation}\label{eq:MSSIM}
    \textrm{MSSIM}(X,Y) = \frac{1}{M}\sum_{j=1}^M \textrm{SSIM}(x_j,y_j)
\end{equation}
که در آن $X$ تصویر مرجع، $Y$ تصویر تخریب شده؛ $x_j$ و $y_j$ اجزاء $j$امین پنجره در تصاویر و $M$، تعداد پنجره‌ها می‌باشد. 
$\textrm{SSIM}(x,y)$ مطابق زیر تعریف می‏‌شود:
\begin{equation}\label{eq:SSIM}
    \textrm{SSIM}(x,y)=\frac{(2\mu_x\mu_y+C_1)(2\sigma_{xy}+C_2)}{(\mu_x^2+\mu_y^2+C_1)(\sigma_x^2+\sigma_y^2+C_2)}
\end{equation}
که در آن $C_1,C_2$ ثوابتی برای پایداری و \lr{$\mu_x, \sigma_x$, $\sigma_{xy}$} تخمین آمارگان محلی تصویر هستند که در \cite{Wang04image} تعریف شده‌اند. 


\subsection{لحاظ کردن \lr{SSIM} در الگوریتم لوکاس-کاناد}
$\textrm{MSSIM}(X,Y)$ به نحوی تعریف شده است که هر چه دو تصویر به هم شبیه‌تر باشند این معیار به 1 نزدیک‌تر خواهد بود. اما ما در اینجا به معیاری نیاز داریم که میزان تفاوت دو تصویر را نشان دهد. به این منظور از $-\textrm{SSIM}$ استفاده نموده و آنرا \lr{SDIS}\LTRfootnote{ Structural DISsimilarity} می‌نامیم:
\begin{equation}\label{eq:SDISneg}
    \textrm{SDIS}(x,y) = -\textrm{SSIM}(x,y)
\end{equation}
بر اساس این تعریف، تفاوت بیشتر دو تصویر مقدار بزرگتری از \lr{SDIS} را نتیجه خواهد داد.
\lr{SSIM} بین پیکسلهای متناظر دو تصویر تعریف می‌شود؛ تصویری که از مقایسه‌ی شباهت تک‌ تک پیکسلهای دو تصویر با این معیار حاصل می‌شود در \cite{Wang04image}، \lr{SSIM~map~image} نامیده شده است، به صورت متناظر در اینجا تصویری را که از مقایسه‌ی تفاوت دو تصویر بر اساس \eqref{eq:SDISneg} ایجاد می‌شود \lr{SDIS~map~image} می‌نامیم. از آنجا که در ادامه از این معیار به عنوان میزان خطا در ثبت تصویر استفاده خواهیم کرد آنرا با $E_{\textrm{SDIS}}$ نشان می‌دهیم.  با درنظر گرفتن این معیار به عنوان ضریبی از میانگین مربعات خطا، رابطه‌ی \eqref{eq:SSD_L2Norm} به صورت زیر در خواهد آمد:
\begin{equation}\label{eq:SSD_L2Norm_SDIS}
    \sum_x E_{\textrm{SDIS}}.[\IWarp-T(\mathbf{x})]^2
\end{equation}
که در آن منظور از نقطه، ضرب عناصر نظیر در دو ماتریس است. برای کمینه‌سازی \eqref{eq:SSD_L2Norm_SDIS}، با یک شیوه‌ی تکراری مشابه \eqref{eq:SSD_L2Norm_deltap} بایستی تابع زیر را کمینه نماییم:
\begin{equation}\label{eq:SSD_SDIS_deltap}
    \sum_x E_{\textrm{SDIS}}.[I(\mathbf{W}(\mathbf{x;\mathbf{p+\triangle p}}))-T(\mathbf{x})]^2
\end{equation}
که در آن $E_{\textrm{SDIS}}$ در $\warp$ ارزیابی می‌شود. با انجام بسط تیلور مرتبه‌ی اول روی $I(\mathbf{W}(\mathbf{x;\mathbf{p+\triangle p}}))$ داریم:
\begin{align}
    &\textrm{SSD} =\label{eq:SSD_SDIS_Taylor} \\
	&\sum_x E_{\textrm{SDIS}}.[\IWarp+\nabla I\roundB{W}{p}\triangle \mathbf{p}-T(\mathbf{x})]^2 \nonumber
\end{align}
که در آن:
$\nabla I=(\round{I}{x},\round{I}{y})$ گرادیان تصویر $I$، ارزیابی شده در $\warp$ و $\roundB{W}{p}$ ژاکوبین مدل تبدیل می‌باشد.

از ادامه مطلب صرفنظر می‌کنیم.
\begin{figure}[tp]
\centering \subfigure[{تصویر با وضوح پایین ورودی}]{\label{fig:Results:LR}
\includegraphics*[width = .23\columnwidth]{Images/Kamandar_LR_Nearest.jpg}}%\vspace{2mm}
\subfigure[{تصویر آموزشی با وضوح بالا}]{\label{fig:Results:HR}
\includegraphics*[width = .23\columnwidth]{Images/Neyzeh_dar.jpg}}%\hspace{2mm}
\subfigure[{نتیجه نهایی در \lr{SNR=90dB}}]{\label{fig:Results:Final}
\includegraphics*[width = .46\columnwidth]{Images/Kamandar_blendXFus_FA-LM-SSIM.jpg}}
 \caption{نتیجه نهایی افزایش وضوح تصویر ورودی(ا) با استفاده از تصویر (ب) و با روش پیشنهادی در شکل \ref{fig:OneLR_oneHR} که دقیق‌تر نمودن ثبت تصویر در آن با الگوریتم2 و همرنگ نمودن بدون درز با شیوه‌ی ارائه شده در \cite{Burt83multiresolution} انجام شده است.}
\label{fig:Results}
\end{figure}

\section{نتایج پیاده‌سازی}\label{Sec:ExperimentalResults}
شیوه‌ی پیشنهادی  با شیوه‌ی اصلی لوکاس-کاناد~\cite{Lucas81iterative} در الگوریتم1 (\lr{LK-GN}) و شیوه‌ی لوکاس-کاناد با روش بهینه‌سازی لونبرگ-مارکورت\cite{Szeliski96video} (\lr{LK-LM})، از نظر میانگین تعداد تکرار تا همگرائی و میانگین خطا (\LTRfootnote{ Root Mean Square}\lr{RMS}) و در مقادیر مختلف نویز مقایسه شده است. تصاویر مورد استفاده در شکل‌های \ref{fig:Results:LR} و \ref{fig:Results:HR} نشان داده شده‌اند. این تصاویر از یکی از سی‌دی‌های مربوط به نقش برجسته‌ی داریوش در بیستون اخذ شده‌اند. همانگونه که در شکل \ref{fig:Results} مشاهده می‌شود دو تصویر از نظر وضوح، شدت روشنایی و رنگ‌بندی با یکدیگر متفاوت هستند. تفاوت زاویه‌ی اخذ دو تصویر نیز در هنگام نگاشت پروجکتیو تصویر \ref{fig:Results:HR} بر روی تصویر \ref{fig:Results:LR} -که در اینجا نشان داده نشده است- مشخص می‌باشد.

 هدف اصلی بالابردن وضوح قسمت متناظر با تصویر \ref{fig:Results:HR} در تصویر \ref{fig:Results:LR} با شیوه‌ی نشان داده شده در شکل \ref{fig:OneLR_oneHR} است. در مقایسات انجام شده، تمام مراحل شکل \ref{fig:OneLR_oneHR} به استثنای مرحله‌ی «دقیق‌تر نمودن مدل با استفاده از ثبت تصویر مبتنی بر ناحیه» یکسان بوده است. نقطه‌ی آغازین بهینه‌سازی در هر سه الگوریتم، تخمین ماتریس تبدیل بدست آمده در مرحله‌ی قبل با استفاده از الگوریتم \lr{RANSAC} می‌باشد. 
%\newpage
ماهیت تصادفی الگوریتم \lr{RANSAC} موجب می‌شود که در هر اجرا تخمینی متفاوت با اجرای دیگر داشته باشیم. لذا هر 
%اجرای الگوریتم شکل \ref{fig:OneLR_oneHR} 
آزمایش را می‌توان جدا از دیگری دانست. 


\subsection{نتایج مقایسه‌ای ثبت تصویر}
هر سه شیوه‌ی فوق‌الذکر برای تصاویر شکل\ref{fig:Results} و در نرخ سیگنال به نویز\LTRfootnote{ Signal to Noise Ratio (SNR)} برابر با ۱۰، ۳۰، ۵۰، ۷۰ و ۹۰ \lr{dB} از تصویر با وضوح پایین اجرا شده‌اند. هر الگوریتم در هر \lr{SNR} 20 مرتبه اجرا شده است. 

شکل \ref{fig:Conv_Iter} میانگین تعداد تکرارها تا همگرا شدن را برای هر سه روش فوق و در مقادیر مختلف نویز نشان می‌دهد. در هیچ یک از آزمایشات روی این تصاویر، روش \lr{LK} واگرا نشده بود.

\begin{figure}[tp]
\centering
\includegraphics*[height=50mm, width = 0.7\columnwidth]{Images/Afsaran_mean_conv_iter.eps}
\caption{ میانگین تعداد تکرار مورد نیاز تا همگرائی.}
\label{fig:Conv_Iter}
\end{figure}

\subsection{کاربرد در وضوح برتر}
کیفیت بصری تصویر نهائی تولید شده، لازمه‌ی اعتبارسنجی هر الگوریتم \SR{} است. 
شکل \ref{fig:Results:Final} نتیجه‌ی نهائی افزایش وضوح تصویر \ref{fig:Results:LR} با استفاده از تصویر آموزشی \ref{fig:Results:HR} را
 نشان می‌دهد. ضریب بزرگ‌نمایی، 2 در نظر گرفته شده است. 
برای مقایسه‌ چند شیوه‌ی دیگر پیاده‌سازی شده‌اند. مقایسه‌ی تصاویر شکل \ref{fig:Katibeh} کیفیت برتر شیوه‌ی پیشنهادی را به خوبی نشان می‌دهد. به عنوان روش \fusion{} در روش پیشنهادی در این مقاله و روش ارائه شده در \cite{Amintoosi08reconstruction} از تبدیل موجک دوبیشز%\LTRfootnote{ Daubechies}
 و با 3 سطح استفاده شده است. روش مبتنی بر مثال ارائه شده در \cite{Freeman02example} نیز به منظور مقایسه پیاده‌سازی شده و برای حفظ سازگاری بلوکهای مجاور از شیوه‌ی پویش سطر به سطر ذکر شده در همان مرجع استفاده شده است. روشهای افزایش اندازه‌ی تصویرِ \lr{Replication} و \lr{Bicubic} در واقع جزو روشهای افزایش وضوح به حساب نمی‌آیند و نتایج آنها صرفاً برای مقایسه آمده است. ناپدید شدن درز در نواحی مرزی و دقیق‌تر بودن نگاشت در شیوه‌ی پیشنهادی مشخص است.

\begin{figure}[t!]
\centering 
\subfigure[روش بزرگنمائی \lr{Replication}]{ \label{fig:Katibeh:Nearest}
\includegraphics[width=35mm]{Images/Kamandar_LR_Nearest_Cropped.jpg}}
\hspace{2mm}
\subfigure[روش بزرگنمائی \lr{Bicubic}]{ \label{fig:Katibeh:Bicubic}
\includegraphics[width=35mm]{Images/Kamandar_LR_Cropped.jpg}}
\hspace{2mm}
\subfigure[روش  ارائه شده در \cite{Freeman02example}]{ \label{fig:Katibeh:Freeman}
\includegraphics[width=35mm]{Images/Kamandar_Freeman_Cropped.jpg}}
\hspace{2mm}
\subfigure[روش  ارائه شده در \cite{Amintoosi08reconstruction}]{ \label{fig:Katibeh:Fustruction}
\lr{\includegraphics[width=35mm]{Images/Kamandar_Fustruction_Cropped.jpg}}}
\hspace{2mm}
\subfigure[روش  پیشنهادی در این مقاله بدون مرحله‌ی آمیختن با تبدیل موجک]{\label{fig:Katibeh:AreaBasedXFus}
\includegraphics[width=35mm]{Images/Kamandar_blend_FA-LM-SSIM_Cropped.jpg}}
\hspace{2mm}
\subfigure[روش  پیشنهادی در این مقاله]{\label{fig:Katibeh:AreaBasedXFus}
\includegraphics[width=35mm]{Images/Kamandar_blendXFus_Fa-LM-SSIM_Cropped.jpg}}

\caption{بزرگ شده‌ی قسمتی از نتیجه‌ی اجرای شیوه‌های مختلف برای افزایش وضوح شکل \ref{fig:Results:LR}. دقیق‌تر بودن مدل در شیوه‌ی پیشنهادی نسبت به شیوه‌ی ذکر شده در \cite{Amintoosi08reconstruction} که فاقد ثبت تصویر مبتنی بر ناحیه است از مقایسه‌ی قسمت بالای نیزه در شکلهای (و) و (د) مشخص است.}
\label{fig:Katibeh} %% label for entire figure
\end{figure}


\section{جمع‌بندی}\label{Sec:Conclusion}
نویسندگان در \cite{Amintoosi08reconstruction} شیوه‌ای جدید برای افزایش وضوح یک تصویر با استفاده از یک تصویر آموزشی ارائه نموده بودند که در مقاله‌ی حاضر به رفع مشکلاتی از آن پرداخته شد. استفاده از یک روش ثبت تصویر مبتنی بر ناحیه به منظور دقیق‌تر شدن مدل نگاشت تصاویر و حذف مرزهای تصاویر با یک روش همرنگ‌سازی بدون درز مراحلی هستند که در کار قبلی انجام نشده بودند. نوآوری اصلی این مقاله لحاظ کردن معیار شباهت ساختاری دو تصویر در فرمولبندی شیوه‌ی معروف ثبت تصویر لوکاس-کاناد و استفاده از آن در وضوح برتر می‌باشد. نتایج پیاده‌سازی‌های انجام شده برتری شیوه‌ی ثبت تصویر پیشنهادی و همچنین کارائی آنرا در مسأله‌ی وضوح برتر در مقایسه با برخی از دیگر روشها نشان داده است.

\subsection*{سپاس‌گزاری}
مؤلفین وظیفه‌ی خود می‌دانند که از آقای دکتر \lr{Peter Kovesi} بابت توابع سودمند \lr{MATLAB} 
%\LTRfootnote{ School of Computer Science \& Software Engineering, The University of Western Australia:\hfill http://www.csse.uwa.edu.au/}
و آقایان وفا خلیقی، مصطفی واحدی و دکتر مهدی امیدعلی بابت زحمات و راهنمایی‌های ارزنده‌ی آنها در زمینه‌ی
زی‌پرشین\RTLfootnote{زی‌پرشین با لوگوی \lr{\XePersian} بسته‌ی حروف‌چینی رایگان فارسی مبتنی بر \lr{\LaTeXe} 
و تحت سیستم‌عامل‌های ویندوز، لینوکس و مک
 می‌باشد:
 {\lr{http://www.parsilatex.com/\hfill}}} (که این مقاله با آن آماده شده است) تشکر به عمل آورند. 



%\begin{latin}
{
% سه دستور زیر باعث می‌شوند که مراجع با قلم کوچکتر و با فاصله خطوط کمتر و با فاصله بین مراجع کم ظاهر شوند. 
% این حالت برای کاهش تعداد صفحات مقاله مناسب است.
% می‌توانید هر یک از آنها را comment نموده و خروجی را ملاحظه فرمایید.
\small % این دستور الزامی نیست.
%\singlespacing
%\setlength{\itemsep}{-2ex}
%اگر از فایلهای سبک انگلیسی مانند latex8.bst یا IEEEtrans استفاده کنید بایستی کل قسمت مراجع را در داخل یک محیط latin قرار دهید (که در حال حاضر comment شده است و دستور زیر را نیز از حالت comment خارج کنید.
\renewcommand{\refname}{\rl{مراجع\hfill}}
\bibliographystyle{ieeetr-fa}%{IEEEtrans}%{latex8}%
\bibliography{References}
}
%\end{latin}

\end{document}