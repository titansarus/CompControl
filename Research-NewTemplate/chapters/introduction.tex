

\فصل{مقدمه}

حسگرها، یکی از مهم‌ترین اجزای سیستم‌های کنترل هستند. به طور کلی حسگرها وسایلی هستند که وظیفه شناسایی رویدادها و تغییرات در مورد یک یا چند متغیر و ارسال آن به در قالب سیگنال‌هایی به سایر وسایل را دارند.

سنسورها به شکل‌های مختلفی دسته‌بندی می‌شوند. یکی از رایج‌ترین انواع دسته‌بندی، تقسیم‌بندی آنان مبتنی بر تماس یا عدم تماس با جسمی است که قصد اندازه‌گیری کمیت در مورد آن وجود دارد. بر این اساس دو نوع حسگر وجود دارد؛ حسگر تماسی و حسگر غیرتماسی.

حسگر‌های تماسی، برای اندازه‌گیری کمیت موردنظر، باید تماس مستقیمی با جسم مدنظر داشته باشند. مثلا بسیاری از حسگر‌های مکانیکی به این شکل فعالیت کرده و از طریق تماس فنر یا اجزای مکانیکی دیگر، با جسم مدنظر تماس برقرار می‌کنند.

حسگر‌های غیرتماسی، برای اندازه‌گیری کمیت موردنظر، نیاز به تماس مستقیم ندارند و از طرق دیگر اندازه‌گیری مدنظر را انجام می‌دهند. حسگر‌های نوری، رادیویی و صوتی از این دسته‌اند.

حسگرهای مبتنی بر اثر هال، نوعی حسگر غیرتماسی است که براساس خاصیت مغناطیسی القا شده در رسانا و ولتاژ ناشی از آن کار می‌کند. با توجه به این موضوع، این حسگر می‌تواند بدون تماس مستقیم، اندازه‌گیری‌‌های لازم را انجام بدهد و با توجه به استهلاک بسیار کم خود حسگر به دلیل عدم برخورد فیزیکی، در محیط‌هایی با شرایط فیزیکی سخت هم به کار گرفته می شود.

در این گزارش، ابتدا در مورد حسگر‌های تماسی صحبت می‌کنیم. پس از آن اثر هال را به صورت فیزیکی بررسی کرده و سپس به سراغ بررسی حسگرهای مبتنی بر اثر هال می‌رویم. در نهایت و در انتها، به بررسی حسگرهای صنعتی مورد استفاده می‌پردازیم. به طور خاص یک حسگر ساخته شرکت
\lr{Diodes Incorporated}
را بررسی کرده و به طور خلاصه به فناوری مورد استفاده در حسگر‌های شرکت \lr{Hioki} هم اشاره خواهیم کرد.

