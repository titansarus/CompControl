
\فصل{نتیجه‌گیری}


در این گزارش، به حسگر‌های غیرتماسی مبتنی بر اثرهال پرداختیم. در ابتدا به طور کلی حسگرهای غیرتماسی و دلایل استفاده از آنان بررسی شد. پس از آنان به توضیح تئوری اثر هال پرداخته و در قالب روابط شناخته شده فیزیک، علت وقوع این پدیده توضیح داده شد.  پس از آن حسگرهای مبتنی بر این اثر بررسی شدند. در ابتدا نحوه ساخت و طراحی آنان مورد بررسی قرار گرفت.  پس از آن نحوه کارکرد این قطعات به عنوان حسگر به شکل‌های مختلف مورد مطالعه قرار گرفته و پس از آن کاربردهای واقعی این حسگرها در وسایل روزمره مورد ارزیابی قرار گرفت. در نهایت به صورت مختصر به بیان توضیحاتی پیرامون قطعه صنعتی \lr{AH266}  که یک حسگر مبتنی بر اثر هال است و فناوری مورد استفاده در حسگر‌های جریان شرکت \lr{Hioki} که در برخی از آن‌ها از از عناصر مبتنی بر اثر‌هال استفاده شده، پرداختیم.
