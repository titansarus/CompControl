

\فصل{برخی نکات نگارشی}

این فصل حاوی برخی نکات ابتدایی ولی بسیار مهم در نگارش متون فارسی است. 
نکات گردآوری‌شده در این فصل به‌ هیچ‌ وجه کامل نیست، 
ولی دربردارنده‌ی حداقل مواردی است که رعایت آن‌ها در نگارش پایان‌نامه ضروری به نظر می‌رسد.

\قسمت{فاصله‌گذاری}

\شروع{شمارش}

\فقره 
علائم سجاوندی مانند نقطه، ویرگول، دونقطه، نقطه‌ویرگول، علامت سؤال، و علامت تعجب (. ، : ؛ ؟ !) بدون فاصله از کلمه‌ی پیشین خود نوشته می‌شوند، ولی بعد از آن‌ها باید یک فاصله‌ قرار گیرد. مانند: من، تو، او.
\فقره 
علامت‌های پرانتز، آکولاد، کروشه، نقل قول و نظایر آن‌ها بدون فاصله با عبارات داخل خود نوشته می‌شوند، ولی با عبارات اطراف خود یک فاصله دارند. مانند: (این عبارت) یا {آن عبارت}.
\فقره 
دو کلمه‌ی متوالی در یک جمله همواره با یک فاصله از هم جدا می‌شوند، ولی اجزای یک کلمه‌ی مرکب باید با نیم‌فاصله\زیرنویس{«نیم‌فاصله» فاصله‌‌ای مجازی است که در عین جدا کردن اجزای یک کلمه‌ی مرکب از یک‌دیگر، آن‌ها را نزدیک به هم نگه می‌دارد. معمولاً برای تولید این نوع فاصله در صفحه‌کلید‌های استاندارد از ترکیب Shift+Space استفاده می‌شود.}‌‌
 از هم جدا شوند. مانند: کلاسِ درس، محبت‌آمیز، دوبخشی.
\پایان{شمارش}


\قسمت{شکل حروف}

\شروع{شمارش}

\فقره 
در متون فارسی به جای حروف «ك» و «ي» عربی باید از حروف «ک» و «ی» فارسی استفاده شود. همچنین به جای اعداد عربی مانند ٥ و ٦ باید از اعداد فارسی مانند ۵ و ۶ استفاده نمود. 
برای این کار، توصیه می‌شود صفحه‌کلید‌ فارسی استاندارد\زیرنویس{\href{http://persian-computing.ir/download/Iranian_Standard_Persian_Keyboard_(ISIRI_9147)_(Version_2.0).zip}{صفحه‌کلید فارسی استاندارد برای ویندوز}، تهیه‌شده توسط بهنام اسفهبد} را بر روی سیستم خود نصب کنید.
\فقره 
عبارات نقل‌قول‌شده یا مؤکد باید درون علامت نقل قولِ «» قرار گیرند، نه ''``. مانند: «کشور ایران».
\فقره 
کسره‌ی اضافه‌ی بعد از «ه» غیرملفوظ به صورت «ه‌ی» نوشته می‌شود، نه «هٔ». مانند: خانه‌ی علی، دنباله‌ی فیبوناچی.

        تبصره‌: اگر «ه» ملفوظ باشد، نیاز به «‌ی» ندارد. مانند: فرمانده دلیر، پادشه خوبان. 

\فقره 
پایه‌های همزه در کلمات، همیشه «ئـ» است، مانند: مسئله و مسئول، مگر در مواردی که همزه ساکن است که در این ‌صورت باید متناسب با اعراب حرف پیش از خود نوشته شود. مانند: رأس، مؤمن. 

\پایان{شمارش}


\قسمت{جدانویسی}

\شروع{شمارش}

\فقره 
اجزای فعل‌های مرکب با فاصله از یک‌دیگر نوشته می‌شوند، مانند: تحریر کردن، به سر آمدن.
\فقره 
علامت استمرار، «می»، توسط نیم‌فاصله از جزء‌ بعدی فعل جدا می‌شود. مانند: می‌رود، می‌توانیم.
\فقره 
شناسه‌های «ام»، «ای»، «ایم»، «اید» و «اند» توسط نیم‌فاصله، و شناسه‌ی «است» توسط فاصله از کلمه‌ی پیش از خود جدا می‌شوند. مانند: گفته‌ام، گفته‌ای، گفته است.
\فقره 
علامت جمع «ها» توسط نیم‌فاصله از کلمه‌ی پیش از خود جدا می‌شود. مانند: این‌ها، کتاب‌ها.
\فقره 
«به» همیشه جدا از کلمه‌ی بعد از خود نوشته می‌شود، مانند: به‌ نام و به آن‌ها، مگر در مواردی که «بـ» صفت یا فعل ساخته است. مانند: بسزا، ببینم.
\فقره 
«به» همواره با فاصله از کلمه‌ی بعد از خود نوشته می‌شود، مگر در مواردی که «به» جزئی از یک اسم یا صفت مرکب است. مانند: تناظر یک‌به‌یک، سفر به تاریخ. 
\پایان{شمارش}


\قسمت{جدانویسی مرجح}

\شروع{شمارش}

\فقره 
اجزای اسم‌ها، صفت‌ها، و قیدهای مرکب توسط نیم‌فاصله از یک‌دیگر جدا می‌شوند. مانند: دانش‌جو، کتاب‌خانه، گفت‌وگو، آن‌گاه، دل‌پذیر.

        تبصره: اجزای منتهی به «هاء ملفوظ» را می‌توان از این قانون مستثنی کرد. مانند: راهنما، رهبر. 

\فقره 
علامت صفت برتری، «تر»، و علامت صفت برترین، «ترین»، توسط نیم‌فاصله از کلمه‌ی پیش از خود جدا می‌شوند. مانند: بیش‌تر، کم‌ترین.

        تبصره‌: کلمات «بهتر» و «بهترین» را می‌توان از این قاعده مستثنی نمود. 

\فقره 
پیشوندها و پسوندهای جامد، چسبیده به کلمه‌ی پیش یا پس از خود نوشته می‌شوند. مانند: همسر، دانشکده، دانشگاه.

        تبصره‌: در مواردی که خواندن کلمه دچار اشکال می‌شود، می‌توان پسوند یا پیشوند را جدا کرد. مانند: هم‌میهن، هم‌ارزی. 

\فقره 
ضمیرهای متصل چسبیده به کلمه‌ی پیش‌ از خود نوشته می‌شوند. مانند: کتابم، نامت، کلامشان. 

\پایان{شمارش}

